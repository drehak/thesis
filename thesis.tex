%%%%%%%%%%%%%%%%%%%%%%%%%%%%%%%%%%%%%%%%%%%%%%%%%%%%%%%%%%%%%%%%%%%%
%% I, the copyright holder of this work, release this work into the
%% public domain. This applies worldwide. In some countries this may
%% not be legally possible; if so: I grant anyone the right to use
%% this work for any purpose, without any conditions, unless such
%% conditions are required by law.
%%%%%%%%%%%%%%%%%%%%%%%%%%%%%%%%%%%%%%%%%%%%%%%%%%%%%%%%%%%%%%%%%%%%

\documentclass[
  digital,     %% The `digital` option enables the default options for the
               %% digital version of a document. Replace with `printed`
               %% to enable the default options for the printed version
               %% of a document.
%%  color,       %% Uncomment these lines (by removing the %% at the
%%               %% beginning) to use color in the printed version of your
%%               %% document
  oneside,     %% The `oneside` option enables one-sided typesetting,
               %% which is preferred if you are only going to submit a
               %% digital version of your thesis. Replace with `twoside`
               %% for double-sided typesetting if you are planning to
               %% also print your thesis. For double-sided typesetting,
               %% use at least 120 g/m² paper to prevent show-through.
  nosansbold,  %% The `nosansbold` option prevents the use of the
               %% sans-serif type face for bold text. Replace with
               %% `sansbold` to use sans-serif type face for bold text.
  nocolorbold, %% The `nocolorbold` option disables the usage of the
               %% blue color for bold text, instead using black. Replace
               %% with `colorbold` to use blue for bold text.
  lof,         %% The `lof` option prints the List of Figures. Replace
               %% with `nolof` to hide the List of Figures.
  lot,         %% The `lot` option prints the List of Tables. Replace
               %% with `nolot` to hide the List of Tables.
]{fithesis4}
%% The following section sets up the locales used in the thesis.
\thesissetup{
    date        = \the\year/\the\month/\the\day,
    university  = mu,
    faculty     = fi,
    type        = bc,
    department  = Department of Machine Learning and Data Processing,
    author      = Dominik Rehák,
    gender      = m,
    advisor     = {RNDr. Vít Novotný, Ph.D.},
    title       = {The Proof of P = NP},
    TeXtitle    = {The Proof of $\mathsf{P}=\mathsf{NP}$},
    keywords    = {keyword1, keyword2, ...},
    TeXkeywords = {keyword1, keyword2, \ldots},
    abstract    = {%
      This is the abstract of my thesis, which can

      span multiple paragraphs.
    },
    thanks      = {%
      These are the acknowledgements for my thesis, which can

      span multiple paragraphs.
    },
    bib         = bibliography.bib,
}
\usepackage{makeidx}      %% The `makeidx` package contains
\makeindex                %% helper commands for index typesetting.
\usepackage[acronym]{glossaries}          %% The `glossaries` package
\renewcommand*\glspostdescription{\hfill} %% contains helper commands
\loadglsentries{terms-abbrs.tex}          %% for typesetting glossaries
\makenoidxglossaries                      %% and lists of abbreviations.
%% These additional packages are used within the document:
\usepackage{paralist} %% Compact list environments
\usepackage{amsmath}  %% Mathematics
\usepackage{amsthm}
\usepackage{amsfonts}
\usepackage{url}      %% Hyperlinks
\usepackage{markdown} %% Lightweight markup
\usepackage{listings} %% Source code highlighting
\lstset{
  basicstyle      = \ttfamily,
  identifierstyle = \color{black},
  keywordstyle    = \color{blue},
  keywordstyle    = {[2]\color{cyan}},
  keywordstyle    = {[3]\color{olive}},
  stringstyle     = \color{teal},
  commentstyle    = \itshape\color{magenta},
  breaklines      = true,
}
\usepackage{floatrow} %% Putting captions above tables
\floatsetup[table]{capposition=top}
\usepackage[babel]{csquotes} %% Context-sensitive quotation marks
\begin{document}
%% Uncomment the following lines (by removing the %% at the beginning)
%% and to print out List of Abbreviations and/or Glossary in your
%% document. Titles for these tables can be changed by replacing the
%% titles `Abbreviations` and `Glossary`, respectively.
%% \clearpage
%% \printnoidxglossary[title={Abbreviations}, type=\acronymtype]
%% \printnoidxglossary[title={Glossary}]

%% The \chapter* command can be used to produce unnumbered chapters:
\chapter*{Introduction}
%% Unlike \chapter, \chapter* does not update the headings and does not
%% enter the chapter to the table of contents. I we want correct
%% headings and a table of contents entry, we must add them manually:
\markright{\textsc{Introduction}}
\addcontentsline{toc}{chapter}{Introduction}

\chapter{Using lightweight markup}
\begin{markdown*}{%
  hybrid,
  definitionLists,
  footnotes,
  inlineFootnotes,
  hashEnumerators,
  fencedCode,
  citations,
  citationNbsps,
  pipeTables,
  tableCaptions,
}

If you decide that \LaTeX{} is too wordy for some parts of your
document, there are [packages](https://www.ctan.org/pkg/markdown
"Markdown") that allow you to use more lightweight markup next
to it.

 ![logo](fithesis/logo/mu/fithesis-fi-color "The old-style seal of
         the Faculty of Informatics")

| Right | Left | Default | Center |
|------:|:-----|---------|:------:|
|    12 | 12   | 12      |   12   |
|   123 | 123  | 123     |   123  |
|     1 | 1    | 1       |    1   |

: This is a table with different types of alignment.

This is a bullet list. Unlike numbered lists, bulleted lists
contain an **unordered** set of bullet points. When a bullet point
contains multiple paragraphs, the list is typeset as follows:

  * The first item of a bullet list

    that spans several paragraphs,
  * the second item of a bullet list,
  * the third item of a bullet list.

When none of the bullet points contains multiple paragraphs, the
list has a more compact form:

  * The first item of a bullet list,
  * the second item of a bullet list,
  * the third item of a bullet list.

Unlike a bulleted list, a numbered list implies chronology or
ordering of the bullet points. When a bullet point
contains multiple paragraphs, the list is typeset as follows:

  1. The first item of an ordered list

     that spans several paragraphs,
  2. the second item of an ordered list,
  3. the third item of an ordered list.
  #. If you are feeling lazy,
  #. you can use hash enumerators as well.

When none of the bullet points contains multiple paragraphs, the
list has a more compact form:

  6. The first item of an ordered list,
  7. the second item of an ordered list,
  8. the third item of an ordered list.

Definition lists are used to provide definitions of terms. When
a definition contains multiple paragraphs, the list is typeset
as follows:

Term 1

:   Definition 1

*Term 2*

:   Definition 2

        Some code, part of Definition 2

    Third paragraph of Definition 2.

When none of the bullet points contains multiple paragraphs, the
list has a more compact form:

Term 1
:   Definition 1
*Term 2*
:   Definition 2

Block quotations are used to include an excerpt from an external
document in way that visually clearly separates the excerpt from
the rest of the work:

> This is the first level of quoting.
>
> > This is nested blockquote.
>
> Back to the first level.

Footnotes are used to include additional information to the
document that are not necessary for the understanding of the main
text. Here is a footnote reference^[Here is the footnote.] and
another.[^longnote]

[^longnote]: Here's one with multiple blocks.

    Subsequent paragraphs are indented to show that they
belong to the previous footnote.

        Some code

    The whole paragraph can be indented, or just the first
    line.  In this way, multi-paragraph footnotes work like
    multi-paragraph list items.

Citations are used to provide bibliographical references to other
documents. This is a regular citation~[@borgman03, p. 123]. This is
an in-text citation: @borgman03\. You can also cite several authors
at once using both regular~[see @borgman03, p. 123; @greenberg98,
sec.  3.2; and @thanh01] and in-text citations: @borgman03 [p.123;
@greenberg98, sec. 3.2; @thanh01].

Code blocks are used to include source code listings into the
document:

    #include <stdio.h>
    #include <unistd.h>
    #include <sys/types.h>
    #include <sys/wait.h>
    // This is a comment
    int main(int argc, char **argv)
    {
        while (--c > 1 && !fork());
        sleep(c = atoi(v[c]));
        printf("%d\n", c);
        wait(0);
        return 0;
    }

There is an alternative syntax for code blocks that allows you to
specify additional information, such as the language of the source
code. This information can be used for syntax highlighting:

``` sh
#!/bin/sh
fac() {
  if [ "$1" -leq 1 ]; then
    echo 1
  else
    echo $(("$1" * fac $(("$1" - 1))))
  fi
}
``````````````

~~~~~~ Ruby
# Here's a way to empty an array.
joe = [ 'eggs.', 'some', 'break', 'to', 'Have' ]
print(joe.pop, " ") while joe.size > 0
print "\n"
~~~~~~

\end{markdown*}

\chapter{These are}
\section{the available}
\subsection{sectioning}
\subsubsection{commands.}
\paragraph{Paragraphs and}
\subparagraph{subparagraphs are available as well.}
Inside the text, you can also use unnumbered lists,
\begin{itemize}
  \item such as
  \item this one
  \begin{itemize}
    \item     and they can be nested as well.
    \item[>>] You can even turn the bullets into something fancier,
    \item[\S] if you so desire.
  \end{itemize}
\end{itemize}
Numbered lists are
\begin{enumerate}
  \item very
  \begin{enumerate}
    \item similar
  \end{enumerate}
\end{enumerate}
and so are description lists:
\begin{description}
  \item[Description list]
    A list of terms with a description of each term
\end{description}
The spacing of these lists is geared towards paragraphs of text.
For lists of words and phrases, the \textsf{paralist} package
offers commands
\begin{compactitem}
  \item that
  \begin{compactitem}
    \item are
    \begin{compactitem}
      \item better
      \begin{compactitem}
        \item suited
      \end{compactitem}
    \end{compactitem}
  \end{compactitem}
\end{compactitem}
\begin{compactenum}
  \item to
  \begin{compactenum}
    \item this
    \begin{compactenum}
      \item kind of
      \begin{compactenum}
        \item content.
      \end{compactenum}
    \end{compactenum}
  \end{compactenum}
\end{compactenum}
The \textsf{amsthm} package provides the commands necessary for the
typesetting of mathematical definitions, theorems, lemmas and
proofs.

\chapter{Floats and references}
\begin{figure}
  \begin{center}
    %% PNG and JPG images can be inserted into the document as well,
    %% but their resolution needs to be adequate. The minimum is
    %% about 100 pixels per 1 centimeter or 300 pixels per 1 inch.
    %% That means that a JPG or PNG image typeset at 4 × 4 cm should
    %% be 400 × 400 px large at the bare minimum.
    %%
    %% The optimum is about 250 pixels per 1 centimeter or 600
    %% pixels per 1 inch. That means that a JPG or PNG image typeset
    %% at 4 × 4 cm should be 1000 × 1000 px large or larger.
    \includegraphics[width=6.3cm]{fithesis/logo/mu/fithesis-base-english-color}
  \end{center}
  \caption{The logo of \acrlong{MU} at 6.3\,cm}
  \label{fig:mulogo1}
\end{figure}

\begin{figure}
  \begin{center}
    \begin{minipage}{.5\textwidth}
      \includegraphics[width=\textwidth]{fithesis/logo/mu/fithesis-base-english-color}
    \end{minipage}
    \hfill  % Fill the horizontal space between the images
    \begin{minipage}{.33\textwidth}
      \includegraphics[width=\textwidth]{fithesis/logo/mu/fithesis-base-english-color} \\[1em]
      \includegraphics[width=\textwidth]{fithesis/logo/mu/fithesis-base-english-color}
    \end{minipage}
  \end{center}
\caption{The logo of \acrlong{MU} at $\frac12$ and
    $\frac13$ of text width}
  \label{fig:mulogo2}
\end{figure}

\begin{table}
  \begin{tabularx}{\textwidth}{lllX}
    \toprule
    Day & Min Temp & Max Temp & Summary \\
    \midrule
    Monday & $13^{\circ}\mathrm{C}$ & $21^\circ\mathrm{C}$ & A
    clear day with low wind and no adverse current advisories. \\
    Tuesday & $11^{\circ}\mathrm{C}$ & $17^\circ\mathrm{C}$ & A
    trough of low pressure will come from the northwest. \\
    Wednesday & $10^{\circ}\mathrm{C}$ &
    $21^\circ\mathrm{C}$ & Rain will spread to all parts during the
    morning. \\
    \bottomrule
  \end{tabularx}
  \caption{A weather forecast}
  \label{tab:weather}
\end{table}

The logo of \gls{MU} is shown in Figure \ref{fig:mulogo1} and
Figure \ref{fig:mulogo2} at pages \pageref{fig:mulogo1} and
\pageref{fig:mulogo2}. The weather forecast is shown in Table
\ref{tab:weather} at page \pageref{tab:weather}. The following
chapter is Chapter \ref{chap:matheq} and starts at page
\pageref{chap:matheq}.  Items \ref{item:star1}, \ref{item:star2},
and \ref{item:star3} are starred in the following list:
\begin{compactenum}
  \item some text
  \item some other text
  \item $\star$ \label{item:star1}
  \begin{compactenum}
    \item some text
    \item $\star$ \label{item:star2}
    \item some other text
    \begin{compactenum}
      \item some text
      \item some other text
      \item yet another piece of text
      \item $\star$ \label{item:star3}
    \end{compactenum}
    \item yet another piece of text
  \end{compactenum}
  \item yet another piece of text
\end{compactenum}
If your reference points to a place that has not yet been typeset,
the \verb"\ref" command will expand to \textbf{??} during the first
run of
\texttt{lualatex \jobname.tex}
and a second run is going to be needed for the references to
resolve. With online services -- such as \Gls{Overleaf} -- this is
performed automatically.

\chapter{Inserting the bibliography}
After linking a bibliography data\-base files to the document using
the \verb"\"\texttt{thesis\discretionary{-}{}{}setup\{bib\discretionary{=}{=}{=}%
\{\textit{file1},\textit{file2},\,\ldots\,\}\}} command, you can
start citing the entries. This is just dummy text
\parencite{borgman03} lightly sprinkled with citations
\parencite[p.~123]{greenberg98}. Several sources can be cited at
once: \cite{borgman03,greenberg98,thanh01}.
\citetitle{greenberg98} was written by \citeauthor{greenberg98} in
\citeyear{greenberg98}. We can also produce \textcite{greenberg98}%
\ or %% Let us define a compound command:
\def\citeauthoryear#1{(\textcite{#1},~\citeyear{#1})}%
\citeauthoryear{greenberg98}%
. The full bibliographic citation is:
\emph{\fullcite{greenberg98}}. We can easily insert a bibliographic
citation into the footnote\footfullcite{greenberg98}.

The \verb"\nocite" command will not generate any
output\nocite{muni}, but it will insert its arguments into
the bibliography. The \verb"\nocite{*}" command will insert all the
records in the bibliography database file into the bibliography.
Try uncommenting the command
%% \nocite{*}
and watch the bibliography section come apart at the seams.

When typesetting the document for the first time, citing a
\texttt{work} will expand to [\textbf{work}] and the
\verb"\printbibliography" command will produce no output. It is now
necessary to generate the bibliography by running \texttt{biber
\jobname.bcf} from the command line and then by typesetting the
document again twice. During the first run, the bibliography
section and the citations will be typeset, and in the second run,
the bibliography section will appear in the table of contents.

The \texttt{biber} command needs to be executed from within the
directory, where the \LaTeX\ source file is located. In Windows,
the command line can be opened in a directory by holding down the
\textsf{Shift} key and by clicking the right mouse button while
hovering the cursor over a directory.  Select the \textsf{Open
Command Window Here} option in the context menu that opens shortly
afterwards.

With online services -- such as Overleaf -- or when using an
automatic tool -- such as \LaTeX MK -- all commands are executed
automatically. When you omit the \verb"\printbibliography" command,
its location will be decided by the template.

  \printbibliography[heading=bibintoc] %% Print the bibliography.

\chapter{Inserting the index}
After using the \verb"\makeindex" macro and loading the
\texttt{makeidx} package that provides additional indexing
commands, index entries can be created by issuing the \verb"\index"
command. \index{dummy text|(}It is possible to create ranged index
entries, which will encompass a span of text.\index{dummy text|)}
To insert complex typographic material -- such as $\alpha$
\index{alpha@$\alpha$} or \TeX{} \index{TeX@\TeX} --
into the index, you need to specify a text string, which will
determine how the entry will be sorted. It is also possible to
create hierarchal entries. \index{vehicles!trucks}
\index{vehicles!speed cars}

After typesetting the document, it is necessary to generate the
index by running
\begin{center}%
  \texttt{texindy -I latex -C utf8 -L }$\langle$\textit{locale}%
  $\rangle$\texttt{ \jobname.idx}
\end{center}
from the command line, where $\langle$\textit{locale}$\rangle$
corresponds to the main locale of your thesis -- such as
\texttt{english}, and then typesetting the document again.

The \texttt{texindy} command needs to be executed from within the
directory, where the \LaTeX\ source file is located. In Windows,
the command line can be opened in a directory by holding down the
\textsf{Shift} key and by clicking the right mouse button while
hovering the cursor over a directory. Select the \textsf{Open Command
Window Here} option in the context menu that opens shortly
afterwards.

With online services -- such as Overleaf -- the commands are
executed automatically, although the locale may be erroneously
detected, or the \texttt{makeindex} tool (which is only able to
sort entries that contain digits and letters of the English
alphabet) may be used instead of \texttt{texindy}. In either case,
the index will be ill-sorted.

  \makeatletter\thesis@blocks@clear\makeatother
  \phantomsection %% Print the index and insert it into the
  \addcontentsline{toc}{chapter}{\indexname} %% table of contents.
  \printindex

\appendix %% Start the appendices.
\chapter{An appendix}
Here you can insert the appendices of your thesis.

\end{document}

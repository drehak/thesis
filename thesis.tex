%%%%%%%%%%%%%%%%%%%%%%%%%%%%%%%%%%%%%%%%%%%%%%%%%%%%%%%%%%%%%%%%%%%%
%% I, the copyright holder of this work, release this work into the
%% public domain. This applies worldwide. In some countries this may
%% not be legally possible; if so: I grant anyone the right to use
%% this work for any purpose, without any conditions, unless such
%% conditions are required by law.
%%%%%%%%%%%%%%%%%%%%%%%%%%%%%%%%%%%%%%%%%%%%%%%%%%%%%%%%%%%%%%%%%%%%

\newif\iftodo
% Enable todo notes in the margins.
\todotrue

\documentclass[
  digital,     %% The `digital` option enables the default options for the
               %% digital version of a document. Replace with `printed`
               %% to enable the default options for the printed version
               %% of a document.
%%  color,       %% Uncomment these lines (by removing the %% at the
%%               %% beginning) to use color in the printed version of your
%%               %% document
  oneside,     %% The `oneside` option enables one-sided typesetting,
               %% which is preferred if you are only going to submit a
               %% digital version of your thesis. Replace with `twoside`
               %% for double-sided typesetting if you are planning to
               %% also print your thesis. For double-sided typesetting,
               %% use at least 120 g/m² paper to prevent show-through.
  nosansbold,  %% The `nosansbold` option prevents the use of the
               %% sans-serif type face for bold text. Replace with
               %% `sansbold` to use sans-serif type face for bold text.
  nocolorbold, %% The `nocolorbold` option disables the usage of the
               %% blue color for bold text, instead using black. Replace
               %% with `colorbold` to use blue for bold text.
  lof,         %% The `lof` option prints the List of Figures. Replace
               %% with `nolof` to hide the List of Figures.
  lot,         %% The `lot` option prints the List of Tables. Replace
               %% with `nolot` to hide the List of Tables.
]{fithesis4}
%% The following section sets up the locales used in the thesis.
\thesissetup{
    date        = \the\year/\the\month/\the\day,
    university  = mu,
    faculty     = fi,
    type        = bc,
    department  = Department of Machine Learning and Data Processing,
    author      = Dominik Rehák,
    gender      = m,
    advisor     = {RNDr. Vít Novotný, Ph.D.},
    title       = {Generic TeX Writer for the Pandoc Document Converter},
    TeXtitle    = {Generic \TeX{} Writer for the Pandoc Document Converter},
    keywords    = {LaTeX, Lua, Markdown, Pandoc, TeX, text analysis and parsing},
    TeXkeywords = {\LaTeX{}, Lua, Markdown, Pandoc, \TeX{}, text analysis and parsing},
    abstract    = {%
      This is the abstract of my thesis, which can

      span multiple paragraphs.
    },
    thanks      = {%
      These are the acknowledgements for my thesis, which can

      span multiple paragraphs.
    },
    bib         = bibliography.bib,
    %% Remove the following line to use the JVS 2018 faculty logo.
    facultyLogo = fithesis-fi,
}
\usepackage{makeidx}      %% The `makeidx` package contains
\makeindex                %% helper commands for index typesetting.
\usepackage[acronym]{glossaries}          %% The `glossaries` package
\renewcommand*\glspostdescription{\hfill} %% contains helper commands
\loadglsentries{terms-abbrs.tex}          %% for typesetting glossaries
\makenoidxglossaries                      %% and lists of abbreviations.
%% These additional packages are used within the document:
\usepackage{paralist} %% Compact list environments
\usepackage{amsmath}  %% Mathematics
\usepackage{amsthm}
\usepackage{amsfonts}
\usepackage{url}      %% Hyperlinks
\usepackage{markdown} %% Lightweight markup
\usepackage{multicol}
\usepackage[dvipsnames]{xcolor}
\usepackage{tikz}
\usetikzlibrary{positioning}
\usepackage{listings} %% Source code highlighting
\lstset{
  basicstyle      = \ttfamily,
  identifierstyle = \color{black},
  keywordstyle    = \color{blue},
  keywordstyle    = {[2]\color{cyan}},
  keywordstyle    = {[3]\color{olive}},
  stringstyle     = \color{teal},
  commentstyle    = \itshape\color{magenta},
  breaklines      = true,
  columns         = fullflexible,
}
\usepackage{floatrow} %% Putting captions above tables
\floatsetup[table]{capposition=top}
% \usepackage[babel]{csquotes} %% Context-sensitive quotation marks
\usepackage{pdfpages}
\usepackage{graphicx}
\usepackage[export]{adjustbox}
\iftodo
  \usepackage[backgroundcolor=Apricot,linecolor=Apricot]{todonotes}
\else
  \usepackage[disable]{todonotes}
\fi

\usepackage{microtype}
\usepackage{hyphenat}
\hyphenation{light-weight}
\hyphenation{con-struc-tors}

\newcommand\macro[1]{\texttt{\textbackslash{}{#1}}}
\newcommand\pandoc[1]{\macro{pandoc\-{#1}}}
\newcommand\renderer[1]{\macro{markdown\-Renderer\-{#1}}}
\ExplSyntaxOn
\newcommand\file[1]
  {
    \str_case:nn
      { #1 }
      {
        { p2md.lua } { \texttt{pandoc\hyp{}to\hyp{}markdown.lua} }
        { p2md.tex } { \texttt{pandoc\hyp{}to\hyp{}markdown.tex} }
        { p2md.sty } { \texttt{pandoc\hyp{}to\hyp{}markdown.sty} }
        % TODO add default
      }
  }
\ExplSyntaxOff

\begin{document}
%% Uncomment the following lines (by removing the %% at the beginning)
%% and to print out List of Abbreviations and/or Glossary in your
%% document. Titles for these tables can be changed by replacing the
%% titles `Abbreviations` and `Glossary`, respectively.
%% \clearpage
%% \printnoidxglossary[title={Abbreviations}, type=\acronymtype]
%% \printnoidxglossary[title={Glossary}]

\chapter{Introduction}
\emph{Outline the goals of my thesis and the motivations behind them.}

\chapter{State of the art}
This chapter briefly introduces Pandoc and the relevant parts of its architecture. It also introduces the \TeX{} and \LaTeX{} typesetting systems and finally the Markdown package, which works with both of these systems and which will be utilized to quickly produce a simple proof of concept for the Pandoc writer.

\section{Pandoc}
\emph{Pandoc}~\cite{pandoc} is a utility which can convert between dozens of markup and document formats, such as HTML, docx, various Markdown flavours, some \TeX{} formats (e.g. \LaTeX{} and Con\TeX{}t), or roff macros for Unix manual pages. Pandoc consists of a core library written in the Haskell language and a command line tool providing access to this library.

Internally, the conversion in Pandoc is performed in two phases. First, the input format is converted into a native represenation, also called an \emph{abstract syntax tree} (or AST for short). Then, this AST is converted into the output format. Such a design allows for the input format readers to be written as modules independent of individual output format writers, and vice versa. This also makes it easy to extend the set of Pandoc's supported formats.

It should be noted that a support for a particular format does not have to be bidirectional. For example, Pandoc currently provides a Con\TeX{}t writer, but not a Con\TeX{}t reader. Thus, it is possible to implement a reader without a corresponding writer, and vice versa.

The AST itself is exposed via a special format named \texttt{native}, which inputs and outputs the entire AST in Pandoc's internal Haskell representaion. Alternatively, the format \texttt{json} inputs and outputs the AST in the JSON format.

\subsection{Usage}

The simplest way to use Pandoc is through the provided command line utility. When no arguments are provided, Pandoc takes text formatted with the Markdown markup language from standard input and outputs the HTML equivalent on the standard output:

$ $

\noindent
\begin{lstlisting}
$ echo 'Hello *world*!' | pandoc
<p>Hello <em>world</em>!</p>
\end{lstlisting}

$ $

Other builtin formats for input and output can be specificied using the \texttt{-{}-from} and \texttt{-{}-to} options (\texttt{-f} and \texttt{-t} for short). Naturally, one can also specify names of the input (argument without an option) and output (\texttt{-{}-output} or \texttt{-o}) files. For example, suppose we have the input file \texttt{shopping-list.html} formatted using HTML with the following contents:

$ $

\noindent
\lstset{language=HTML}
\begin{lstlisting}
<h3>Shopping list</h3>
<ul>
    <li>Eggs</li>
    <li>Onions</li>
    <li>Butter</li>
    <li>Bread</li>
</ul>
\end{lstlisting}

$ $

\noindent
Then, the following command will the output file \texttt{shopping-list.1} formatted using macros of the \textit{roff} typesetting system used for Unix manual pages:

$ $

\noindent
\lstset{language=}
\begin{lstlisting}
pandoc -f html -t man shopping-list.html -o shopping-list.1
\end{lstlisting}

$ $

\noindent
The contents of the output file \texttt{shopping-list.1} will be:

$ $

\noindent
\begin{lstlisting}
.SS Shopping list
.IP \[bu] 2
Eggs
.IP \[bu] 2
Onions
.IP \[bu] 2
Butter
\end{lstlisting}
% \] % vim: stop highlighting an equation

$ $

If we want to define our own input/output format, we ultimately have two options. The first one is to directly extend the code of Pandoc itself with a new reader/writer module. This would require at least working knowledge Pandoc's codebase, as well as Haskell. Moreover, such a solution would not be very portable, as to actually use such a module, it would be necessary to recompile Pandoc. Of course, there is the possibility that Pandoc maintainers would consider the module worthy enough and of a sufficient quality (and the format significant enough) to include it into Pandoc itself. Even then, the format would only be available in new versions of Pandoc.

The other option is to write a reader/writer in the Lua programming language using the supported Lua API, which is described in Pandoc's documentation\footnote{\url{https://pandoc.org/custom-writers.html}}. Then, if we simply use the path to the Lua reader/writer as an input/output format, Pandoc's built-in Lua interpreter will parse and use that reader/writer at runtime. For example, if we defined a plain \TeX{} writer in the file \texttt{plaintex.lua} and then wanted to use plain \TeX{} as the output format, it would be enough to call:

$ $

\noindent
\begin{lstlisting}
$ echo 'Hello *world*!' | pandoc -t plaintex.lua
\end{lstlisting}

$ $

\noindent
We will construct such a writer in Chapter 3.

$ $

As an aside, for some output formats (like \LaTeX{}, HTML or roff macros above), the default output produced by Pandoc does not constitute a complete document ready for viewing/typesetting, but rather just a short snippet. This can be amended by the \texttt{-{}-standalone} option (\texttt{-s}~for short), which surrounds the snippet with a template containing the skeleton of a full document. (Most of the templates are too long to reasonably embed an example of such an output here.)

\section{\TeX{}}
\TeX{} is a typesetting system designed and released by Donald Knuth in 1978. One of the main motivations for the creation of \TeX{} was the difficulty surrounding typesetting of mathematics at the time of its inception.\todo{something else?}

The input files of \TeX{} are plain text and their main elements are text to be typeset and control sequences, which influence how the text is to be typeset. 
Control sequences begin with their identifier, which consists of a backslash and a sequence of letters (uppercase or lowercase). Additionally, some control sequences take one or more of following tokens as parameters. The \TeX{} engine itself has around 300 of built-in control sequences, so-called \textit{primitives}.

An example of a primitive is \macro{end}, which is used for halting the \TeX{} engine and which takes no parameters. Another example is the primitive \macro{kern}, which inserts a blank fill of a given length. When the \TeX{} engine encounters \macro{kern1em}, it inserts a blank fill, which is 1em wide.

Besides primitives, there can also be \textit{macros}, which are control sequences that recursively expand into other control sequences and groups of characters. Macros can be defined using the sequence \macro{def}, which can happen directly in the input file, or in a \textit{format}, which is essentially a collection of definitions in a single file. Knuth himself has defined such a format called \textit{plain \TeX{}} as the default format for \TeX{}.

An example of a macro is \macro{quad}, which is defined in plain \TeX{} as follows:

\noindent
\lstset{language=[plain]TeX}
\begin{lstlisting}
\def\enspace{\kern.5em}
\end{lstlisting}

When the macro is defined (e.g. by loading the format) and \macro{quad} is encountered by the \TeX{} engine, it is expanded into \macro{kern.5em}. This results in a 0.5em wide blank fill as outlined above.

Several formats have been created specifically to provide abstractions above plain \TeX{} that allow greater focus on content rather than formatting itself, since plain \TeX{} can be cumbersome for creating documents. The most popular such system is \LaTeX{}, first released in 1984.

There exist multiple compatible implementations of the \TeX{} engine, some of which also implement their own primitives. An example is pdf\TeX{}, which is capable of direct output to PDF and also provides primitives for micro-typography. Another example is Lua\TeX{}, which will be of interest in chapter 3, since it is able to directly execute code written in the Lua language.

\section{The Markdown package for \TeX{}}
The \emph{Markdown}~\cite{cstug-markdown} \TeX{} package was named after Markdown, a lightweight markup language, which was first released in 2004 and designed with the primary goal of being easy to read and write. This has eventually caused wide adoption of the language on the Internet. The Markdown package provides a way to embed blocks of text written in the Markdown language directly into \TeX{} documents. During typesetting, these blocks are parsed and eventually expanded into native \TeX{} macros correspoding to the input. 

Because the Markdown package offers first-class support for plain \TeX{} and already tackles the task of converting high-level Markdown elements into low-level \TeX{} macros, it was suitable for its use in a working prototype of the plain \TeX{} writer for Pandoc. That is, the writer can simply produce macros defined by the Markdown package, which can then be expanded into \TeX{} primitives.

Regarding usage, the main intended way to use the Markdown package with plain \TeX{} is either through the macro \macro{markdownInput}, which is analogous to the \macro{input} primitive and takes a filename as an argument, or the pair of macros \macro{markdownBegin} and \macro{markdownEnd}, which surround a block of Markdown-formatted text directly embedded into the document source. Since we are mainly interested in the underlying macros that concern the individual Markdown element types, the thesis will not mention these macros further.

\subsection{Lunamark writer -- a previously explored approach}
Markdown's parser originates from a Lua library called \emph{Lunamark}~\cite{lunamark}, which is coincidentally also developed by John MacFarlane, the author of Pandoc. Lunamark provides a fast conversion of Markdown (its only input format) into other commonly used markup formats. Since the Lua\TeX{} engine is able to directly execute Lua code, Lunamark's parser was suitable for its use in the Markdown package.

\todo{reformulate, describe this as a standalone project}Here it should be noted that the original goal of the thesis was to integrate the support for Pandoc into the Markdown package. Such support allows users to embed blocks of all input formats, that are supported by Pandoc, into \TeX{} documents. This would be done by extending Lunamark with a reader module for Pandoc's AST (specifically the \texttt{json} format) and this support would then be transferred to the Markdown package. A prototype of this reader was actually written\footnote{\url{https://github.com/drehak/lunamark}}, before we\todo{Who is ``we''? Should I actually mention my supervisor here?} discovered that Pandoc supports external writers. The Lunamark writer was subsequently rewritten as a Pandoc writer, which outputs Markdown's macros, with a shim layer on top. (The details of this are all described in chapter 3). Eventually, it became clear that it is not necessary to extend Markdown with support for Pandoc, and that the macros introduced in what is now the proof of concept for the Pandoc writer could form a standalone package. At that point, the goal of the thesis had to be entirely reformulated.

\section{Comparison of Pandoc and Markdown}

\begin{itemize}
\item they kinda do the same thing
\item common elements
\item elements that work differently
\item elements without equivalents in Markdown
\end{itemize}

Elements of Pandoc's AST are somewhat similar to Markdown's macros for elements. Some of them are simple enough that it is possible to pass their contents to the corresponding macro -- that is, if they even have any.

An example of such an element would be \texttt{Emph}, which directly corresponds to the macro \renderer{Emphasis}. There is a few more elements with such equivalents, although in the case of \texttt{CodeBlock}, the order of its two parameters needs to be swapped:

\begin{itemize}
\item \texttt{Emph} $\rightarrow$ \renderer{Emphasis}
\item \texttt{Strong} $\rightarrow$ \renderer{StrongEmphasis}
\item \texttt{HorizontalRule} $\rightarrow$ \renderer{HorizontalRule}
\item \texttt{LineBreak} $\rightarrow$ \renderer{LineBreak}
\item \texttt{Note} $\rightarrow$ \renderer{Footnote}
\item \texttt{Code} $\rightarrow$ \renderer{CodeSpan}
\item \texttt{CodeBlock} $\rightarrow$ \renderer{InputFencedCode}
\end{itemize}

For other elements, the mapping to Markdown's macros is a bit more complicated.\todo{add more to this}

Some Pandoc elements take a parameter of a type with multiple constant constructors. (In imperative languages, the closest equivalent would be an enumerable type.) For example, one of the parameters of \texttt{Cite} is of the type \texttt{CitationMode}, which itself has three constructors without parameters - \texttt{AuthorInText}, \texttt{SuppressAuthor} and \texttt{NormalCitation}. Another example would be \texttt{MathType} passed as a parameter of \texttt{Math} with two constructors, \texttt{DisplayMath} and \texttt{InlineMath}\footnote{In this case, the API for custom Pandoc writers splits handling of \texttt{Math} into two functions, \texttt{InlineMath} and \texttt{DisplayMath}. Depending on the value of \texttt{MathType}, Pandoc simply calls one of these two functions.}. Similarly, \texttt{Quoted} has a parameter \texttt{QuoteType} with two constructors, \texttt{SingleQuote} and \texttt{DoubleQuote}\footnote{Again, the API splits \texttt{Quoted} into two functions -- \texttt{SingleQuote} and \texttt{DoubleQuote}.} -- although \texttt{Quoted} has no equivalent among Markdown's macros.

Other elements are simply represented differently in Markdown. For example, the element \texttt{Header} contains its contents and an integer representing its level. (This maps nicely to the HTML tags \texttt{<h1>} to \texttt{<h6>}.) Markdown represents this using six individual macros, \renderer{HeadingOne} to \renderer{HeadingSix}.\todo{more groups?}

Finally, there is a group of elements without any corresponding macros, like the afforementioned \texttt{Quoted}. This is understandable, since the syntax of Markdown does not cover these types of elements. A few more trivial examples would be \texttt{Underline}, \texttt{Strikeout}, \texttt{Superscript}, \texttt{Subscript} and \texttt{SmallCaps}.\todo{revise for latest Markdown version ($\rightarrow$ Future work))}

\begin{figure}
  \centering
  \begin{tikzpicture}[->, anchor=base, baseline, sibling distance=2.5cm, level distance=1.35cm]
    \node (pdpara) {\texttt{Para}}
      child { node (pdstrhello) {\texttt{Str "Hello"}} edge from parent[->, gray] }
      child { node (pdspace) {\texttt{Space}} edge from parent[->, gray] }
      child { node (pdemph) {\texttt{Emph}}
        child { node (pdstrmd) {\texttt{Str "Markdown"}} edge from parent[->, gray] }
      edge from parent[->, gray] }
      child { node (exclamation) {\texttt{Str "!"}} edge from parent[->, gray]
    };
  \end{tikzpicture}
  \caption{A graphical representation of Pandoc's AST for the Markdown input ``\texttt{Hello *Markdown*!}''}
  \label{fig:pandoc-ast}
\end{figure}

\begin{figure}
  \centering
  \begin{tikzpicture}[anchor=base, baseline, node distance=0.6cm]
    \node (hello) {Hello\textvisiblespace};
    \node [right=of hello, align=left, xshift=0.2cm] (mdemph) {
      \macro{markdownRenderer} \\
      \texttt{ Emphasis\{Markdown\}}
    } ;
      \node [below=of mdemph, align=left] (mdprotoemph) {
        \macro{markdownRenderer} \\
        \texttt{ EmphasisPrototype} \\
        \texttt{ \{Markdown\}}
      } ;
      \node [below=of mdprotoemph] (texemph) {\macro{emph\{Markdown\}}} ;
      \node [below=of texemph] (emph) {\emph{Markdown}} ;
    \node [right=of mdemph, xshift=0.2cm] (exclamation) {!} ;
    \node [right=of exclamation, align=left, xshift=0.2cm] (mdsep) {
      \macro{markdownRenderer} \\
      \texttt{ InterblockSeparator}
    } ;
      \node [below=of mdsep, align=left] (mdprotosep) {
        \macro{markdownRenderer} \\
        \texttt{ InterblockSeparator} \\
        \texttt{ Prototype}
      } ;
      \node [below=of mdprotosep, yshift=0.14em] (texsep) {\macro{par}} ;

    \path [gray] (hello) edge (mdemph) ;
      \path [gray, ->] (mdemph) edge (mdprotoemph) ;
      \path [gray, ->] (mdprotoemph) edge (texemph) ;
      \path [gray, ->] (texemph) edge (emph) ;
    \path [gray] (mdemph) edge (exclamation) ;
    \path [gray] (exclamation) edge (mdsep) ;
      \path [gray, ->] (mdsep) edge (mdprotosep) ;
      \path [gray, ->] (mdprotosep) edge (texsep) ;
  \end{tikzpicture}
  \caption{A graphical representation of macros generated by the Markdown package for the input ``\texttt{Hello *Markdown*!}''}
  \label{fig:markdown-ast}
\end{figure}

% Prvky Pandocu prevzaté z dokumentácie: \textsf{https://hackage.haskell.org/package/pandoc-types-1.22/docs/Text-Pandoc-Definition.html}

\begin{figure}
  \centering
  \begin{multicols}{3}
    \begin{compactenum}
      \item \texttt{Plain}
      \item \texttt{Para}
      \item \texttt{LineBlock}
      \item \texttt{CodeBlock}
      \item \texttt{RawBlock}
      \item \texttt{BlockQuote}
      \item \texttt{OrderedList}
      \item \texttt{BulletList}
      \item \texttt{DefinitionList}
      \item \texttt{Header}
      \item \texttt{HorizontalRule}
      \item \texttt{Table}
      \item \texttt{Div}
      \item \texttt{Null}
      \item \texttt{Str}
      \item \texttt{Emph}
      \item \texttt{Underline}
      \item \texttt{Strong}
      \item \texttt{Strikeout}
      \item \texttt{Superscript}
      \item \texttt{Subscript}
      \item \texttt{SmallCaps}
      \item \texttt{Quoted}
      \item \texttt{Cite}
      \item \texttt{Code}
      \item \texttt{Space}
      \item \texttt{SoftBreak}
      \item \texttt{LineBreak}
      \item \texttt{Math}
      \item \texttt{RawInline}
      \item \texttt{Link}
      \item \texttt{Image}
      \item \texttt{Note}
      \item \texttt{Span}
    \end{compactenum}
  \end{multicols}
  \vspace*{-1em}
  \caption{Complete list of elements of Pandoc's AST (as of Pandoc~2.14.2)}
  \label{fig:pandoc-elems}
\end{figure}

\begin{figure}
  \centering
  \begin{multicols}{2}
    \footnotesize
    \begin{compactenum}
      \item Tickbox Renderers
      \begin{compactenum}
        \item \texttt{TickedBox}
        \item \texttt{HalfTickedBox}
        \item \texttt{UntickedBox}
      \end{compactenum}
      \item \texttt{InterblockSeparator}
      \item \texttt{LineBreak}
      \item \texttt{Ellipsis}
      \item \texttt{Nbsp}
      \item Special Character Renderers
      \begin{compactenum}
        \item \texttt{Ampersand}
        \item \texttt{Backslash}
        \item \texttt{Circumflex}
        \item \texttt{DollarSign}
        \item \texttt{Hash}
        \item \texttt{LeftBrace}
        \item \texttt{PercentSign}
        \item \texttt{Pipe}
        \item \texttt{RightBrace}
        \item \texttt{Tilde}
        \item \texttt{Underscore}
      \end{compactenum}
      \item \texttt{CodeSpan}
      \item \texttt{Link}
      \item \texttt{Image}
      \item \texttt{ContentBlock}
      \item Bullet List
      \begin{compactenum}
        \item \texttt{UlBegin}
        \item \texttt{UlBeginTight}
        \item \texttt{UlItem}
        \item \texttt{UlItemEnd}
        \item \texttt{UlEnd}
        \item \texttt{UlEndTight}
      \end{compactenum}
      \item Ordered List
      \begin{compactenum}
        \item \texttt{OlBegin}
        \item \texttt{OlBeginTight}
        \item \texttt{OlItem}
        \item \texttt{OlItemEnd}
        \item \texttt{OlItemWithNumber}
        \item \texttt{OlEnd}
        \item \texttt{OlEndTight}
      \end{compactenum}
      \item Definition List
      \begin{compactenum}
        \item \texttt{DlBegin}
        \item \texttt{DlBeginTight}
        \item \texttt{DlItem}
        \item \texttt{DlItemEnd}
        \item \texttt{DlDefinitionBegin}
        \item \texttt{DlDefinitionEnd}
        \item \texttt{DlEnd}
        \item \texttt{DlEndTight}
      \end{compactenum}
      \item Emphasis
      \begin{compactenum}
        \item \texttt{Emphasis}
        \item \texttt{StrongEmphasis}
      \end{compactenum}
      \item Block Quote
      \begin{compactenum}
        \item \texttt{BlockQuoteBegin}
        \item \texttt{BlockQuoteEnd}
      \end{compactenum}
      \item Code Block
      \begin{compactenum}
        \item \texttt{InputVerbatim}
        \item \texttt{InputFencedCode}
      \end{compactenum}
      \item YAML Metadata
      \begin{compactenum}  % nepotrebné?
        \item \texttt{JekyllDataBegin}
        \item \texttt{JekyllDataEnd}
        \item \texttt{JekyllDataMappingBegin}
        \item \texttt{JekyllDataMappingEnd}
        \item \texttt{JekyllDataSequenceBegin}
        \item \texttt{JekyllDataSequenceEnd}
        \item \texttt{JekyllDataBoolean}
        \item \texttt{JekyllDataNumber}
        \item \texttt{JekyllDataString}
        \item \texttt{JekyllDataEmpty}
      \end{compactenum}
      \item Heading
      \begin{compactenum}
        \item \texttt{HeadingOne}
        \item \texttt{HeadingTwo}
        \item \texttt{HeadingThree}
        \item \texttt{HeadingFour}
        \item \texttt{HeadingFive}
        \item \texttt{HeadingSix}
      \end{compactenum}
      \item \texttt{HorizontalRule}
      \item \texttt{Footnote}
      \item \texttt{Cite}
      \item \texttt{TextCite}
      \item \texttt{Table}
      \item \texttt{InlineHtmlComment}
    \end{compactenum}
  \end{multicols}
  \vspace*{-1em}
  \caption{A complete list of output macros of the Markdown package for version 2.11.0}
  \label{fig:markdown-renderers}
\end{figure}

\chapter{Implementation}
In this chapter, I will describe how the proof of concept for the plain \TeX{} writer for Pandoc was implemented. Additionally, I will demonstrate the usage of the writer on a set of example documents.

\section{Plain \TeX{} writer for Pandoc and the macros it outputs}
As described in chapter 2, a custom Pandoc writer consists of a single file written in Lua. The Pandoc documentation provides a sample custom writer for HTML and recommends using this writer as a basis for other custom writers, which is what I did. I will refer to this modified file as \file{p2md.lua}, which is also its name in the proof of concept.

For each AST element type, the writer defines a pure function that outputs the element in the output format. If the element has any child elements, they have already been processed by other functions by the time they are passed as arguments. Therefore, it is not necessary to implement the recursive descent manually.

Functions in \file{p2md.lua} return macros of the form \pandoc{ElementName}, similarly to the Markdown package with its \renderer{ElementName}. These macros are defined inside the \file{p2md.tex} file. This allows the user to change the appearance of elements by simply redefining the macros on the \TeX{} side.\footnote{The proof of concept actually uses this in \file{p2md.sty} to redefine some macros with equivalent \LaTeX{} constructs.} It also results in a somewhat cleaner implementation, since the style of all elements is defined in a single \TeX{} file. \todo{reformulate?}

Internally, Pandoc splits elements into two groups of inline elements and block elements. As the name suggets, block elements appear as vertically separate blocks in the document (e.g. paragraphs, lists or tables), while inline elements can occur inside a line of text - which is generally a block element. This distinction does not matter for the user, perhaps not even the author of a custom writer. However, it constrains which elements can appear in others - for each child element, it is specified whether it's a block element or an inline element. Generally, block elements can contain block elements or inline elements, while inline elements can only contain other inline elements.\footnote{The one exception to this is \texttt{Note}, which represents a footnote or an endnote and its contents are represented as a list of block elements. However, these elements appear outside the parent element of the note, such as the contents of this footnote.}

I have reordered the functions in \file{p2md.lua} so that they appear in the same order in which they are defined in Pandoc\footnote{\url{https://hackage.haskell.org/package/pandoc-types-1.22/docs/Text-Pandoc-Definition.html}}. Nevertheless, since inline elements are structurally simpler and cannot contain block elements, I will describe them first.

\subsection{Inline elements}

The first inline element is \texttt{Str}, which simply defines a string without spaces. A trivial example of this can be seen in figure \ref{fig:html-browsers-typeset}. To provide correct \TeX{} output, it will be necessary to escape characters that have a special role in \TeX{}. Since the Markdown package already provides macros for these characters, for example \renderer{Backslash}, the occurences of these characters will be replaced with their respective macros. (It is possible that letters will occur after the special characters, so braces are added after the macro name, e.g. \pandoc{Backslash\{\}}. This does not affect the behavior of the macro, but it prevents the \TeX{} tokenizer from considering the letters as a part of the macro name.) The sample writer already provides an example function for escaping strings - for HTML, of course:

$ $

\noindent
\lstset{language=[5.3]Lua}
\begin{lstlisting}
local function escape(s, in_attribute)
  return s:gsub('[<>&"\']',
    function(x)
      if x == '<' then
        return '&lt;'
      elseif x == '>' then
        return '&gt;'
      elseif x == '&' then
        return '&amp;'
      elseif in_attribute and x == '"' then
        return '&quot;'
      elseif in_attribute and x == "'" then
        return '&#39;'
      else
        return x
      end
    end)
end
\end{lstlisting}

$ $

\noindent
There is a problem with this construct -- \texttt{gsub} replaces all occurences of the given characters in \texttt{s}. However, if the modified function for plain \TeX{} output would in this way replace \texttt{\{} with \pandoc{LeftBrace\{\}}, this would generate another occurence of \texttt{\{}, which would also get replaced with \pandoc{LeftBrace\{\}}\dots{} and so on. This can be mitigated by saving the substitution into a variable and returning it at the end of the inner function. (Some characters in the pattern given to \texttt{gsub} have to be escaped with \%.)

$ $

\noindent
\lstset{language=[5.3]Lua}
\begin{lstlisting}
local function escape(s)
  s = string.gsub(s, "[\\{}%|_#&~%^%%%$]", function(c)
    local s
    if     c == "&"  then s = "\\pandocAmpersand{}"
    elseif c == "\\" then s = "\\pandocBackslash{}"
    elseif c == "^"  then s = "\\pandocCircumflex{}"
    elseif c == "$"  then s = "\\pandocDollarSign{}"
    elseif c == "#"  then s = "\\pandocHash{}"
    elseif c == "{"  then s = "\\pandocLeftBrace{}"
    elseif c == "%"  then s = "\\pandocPercentSign{}"
    elseif c == "|"  then s = "\\pandocPipe{}"
    elseif c == "}"  then s = "\\pandocRightBrace{}"
    elseif c == "~"  then s = "\\pandocTilde{}"
    elseif c == "_"  then s = "\\pandocUnderscore{}"
    else                  s = c
    end
    return s
  end)
  return s
end
\end{lstlisting}

$ $

\noindent
The function for \texttt{Str} itself is same as in the sample writer.

$ $

\noindent
\lstset{language=[5.3]Lua}
\begin{lstlisting}
function Str(s)
  return escape(s)
end
\end{lstlisting}

$ $

\noindent
Next inline element is \texttt{Emph}, representing emphasis. \texttt{Emph} contains a list of inline elements, but any special characters inside them have already been escaped by \texttt{Str} executed on the strings inside, so there is no need to do that again. Here, the Lua operator \texttt{..} performs concatenation.

$ $

\noindent
\lstset{language=[5.3]Lua}
\begin{lstlisting}
function Emph(s)
  return "\\pandocEmph{" .. s .. "}"
end
\end{lstlisting}

$ $

\noindent
\pandoc{Emph} is defined in \file{p2md.tex}, as outlined above:

$ $

\noindent
\lstset{language=[plain]TeX}
\begin{lstlisting}
\def\pandocEmph{\markdownRendererEmphasis}%
\end{lstlisting}

$ $

\noindent
Next is \texttt{Underline}. The original Markdown language has no way to represent an underline, so there is no support for it in the Markdown package.\footnote{As of version 2.11.0.} For now, we will simply define a macro which expands to its parameter, rendering it as a plain string.

$ $

\noindent
\lstset{language=[5.3]Lua}
\begin{lstlisting}
function Underline(s)
  return "\\pandocUnderline{" .. s .. "}"
end
\end{lstlisting}

$ $

\noindent
\file{p2md.tex}:

$ $

\noindent
\lstset{language=[plain]TeX}
\begin{lstlisting}
\def\pandocUnderline#1{#1}%
\end{lstlisting}

$ $

\noindent
For the next few inline elements, the structure of the writer functions is virtually the same, so it is not worth listing them here. The macros look similar too. The only new construct occurs in \pandoc{SingleQuoted} and \pandoc{DoubleQuoted}, which surround their parameter with \texttt{`} and \texttt{'}, the characters used by \TeX{} to represent quotes:

$ $

\noindent
\lstset{language=[plain]TeX}
\begin{lstlisting}
\def\pandocStrong{\markdownRendererStrongEmphasis}%
\def\pandocStrikeout#1{#1}%
\def\pandocSubscript#1{#1}%
\def\pandocSuperscript#1{#1}%
\def\pandocSmallCaps#1{#1}%
\def\pandocSingleQuoted#1{`#1'}%
\def\pandocDoubleQuoted#1{``#1''}%
\end{lstlisting}

$ $

\noindent
With \texttt{Cite}, things start to get a bit complicated.\todo{describe the Pandoc side} On the side of the Markdown package, the macro \renderer{Cite} takes a variable number of tokens: the first one is the number of citations, followed by a sequence of four tokens -- a flag that indicates whether to suppress displaying the author, then the prefix, suffix and ID of the citation -- for each of those citations. To assemble this in Lua, we construct a simple buffer table\footnote{A table is the basic structure in Lua, which functions as a hash map indexed using numbers. However, Lua provides functions on tables that can make them function more like lists/arrays, such as \texttt{table.insert}, which is analogous to Python's \texttt{append} on lists and inserts a value into the table, with the key being the greatest key in the table increased by 1.}, which we fill using data obtained from the parameters of \texttt{Cite}. Then we return a concatenation of this table:

$ $

\noindent
\lstset{language=[5.3]Lua}
\begin{lstlisting}
function Cite(s, cs)
  local buffer = {"\\pandocCite{" .. #cs .. "}"}
  for _, citation in pairs(cs) do
    if (citation.citationMode == "SuppressAuthor") then
      table.insert(buffer, "{-}")
    else
      table.insert(buffer, "{+}")
    end
    table.insert(buffer, "{" .. citation.citationPrefix .. "}")
    table.insert(buffer, "{" .. citation.citationSuffix .. "}")
    table.insert(buffer, "{" .. citation.citationId .. "}")
  end
  return table.concat(buffer)
end
\end{lstlisting}

$ $

\noindent
\file{p2md.tex}:

$ $

\noindent
\lstset{language=[plain]TeX}
\begin{lstlisting}
\def\pandocCite{\markdownRendererCite}%
\end{lstlisting}

$ $

\noindent
Some of the functions provide a table with attributes in the parameters. This table can get populated by e.g. Pandoc's HTML reader. In the cases of some elements, such as \texttt{Div} or \texttt{CodeBlock} (described in section 3.1.2), one of the attributes usually encodes information necessary for describing the element. As for attributes with arbitrary keys and values, the Markdown package currently has no way to interpret or process them, so the plain \TeX{} writer can discard them for now. Such is the case for \texttt{Code}:

$ $

\noindent
\lstset{language=[5.3]Lua}
\begin{lstlisting}
function Code(s, attr)
  return "\\pandocCode{" .. escape(s) .. "}"
end
\end{lstlisting}

$ $

\noindent
\file{p2md.tex}:

$ $

\noindent
\lstset{language=[plain]TeX}
\begin{lstlisting}
\def\pandocCode{\markdownRendererCodeSpan}%
\end{lstlisting}

$ $

\noindent
Regarding whitespace, the authors of Pandoc have for some reason decided to turn spaces into their own element type. Pandoc also distinguishes between line breaks and so-called ``soft breaks'', which in some formats \textit{suggest} a line break without strictly enforcing it.\todo{check if this is correct} Since neither \TeX{} nor the Markdown package offer a mechanism for this\todo{check too}, it is rather safe to assume a simple space in the place of a soft break too.

Note that there is a new line after \pandoc{LineBreak}. A single newline in itself will not get rendered by \TeX{}. The reason for including it is different -- \TeX{} has a hard memory limit\todo{check limit size} and loading a line which is too long could exhaust this limit. This is why the output document is at some places broken into multiple lines, a pattern which will be utilized again with some block elements.

$ $

\noindent
\lstset{language=[5.3]Lua}
\begin{lstlisting}
function Space()
  return "\\pandocSpace{}"
end

function SoftBreak()
  return "\\pandocSoftBreak{}"
end

function LineBreak()
  return "\\pandocLineBreak\n"
end
\end{lstlisting}

$ $

\noindent
\file{p2md.tex}:

$ $

\noindent
\lstset{language=[plain]TeX}
\begin{lstlisting}
\def\pandocSpace{ }%
\def\pandocSoftBreak{ }%
\def\pandocLineBreak{\markdownRendererLineBreak}%
\end{lstlisting}

$ $

\noindent
As mentioned in section 2.4, the custom writer API splits the element \texttt{Math} into two functions, \texttt{InlineMath} and \texttt{DisplayMath}, the latter of which is technically a block element. In \TeX{}, the former is represented by wrapping the math contents in two dollar characters, while the latter is delimited by \texttt{\textbackslash{}[} and \texttt{\textbackslash{}]}.

Luckily, Pandoc's AST represents all math using \TeX{} syntax for math, so there is no need for additional conversion of math syntax in the plain \TeX{} writer.

$ $

\noindent
\lstset{language=[5.3]Lua}
\begin{lstlisting}
function InlineMath(s)
  return "\\pandocInlineMath{" .. s .. "}"
end

function DisplayMath(s)
  return "\\pandocDisplayMath{" .. s .. "}"
end
\end{lstlisting}

$ $

\noindent
\file{p2md.tex}:

$ $

\noindent
\lstset{language=[plain]TeX}
\begin{lstlisting}
\def\pandocInlineMath#1{$#1$}%
\def\pandocDisplayMath#1{\[#1\]}%
\end{lstlisting}

$ $

\noindent
Some input formats of Pandoc allow raw elements of other format types to be inlined in the output documents. Examples of this are raw HTML inside documents formatted with Markdown, or raw \TeX{} inside HTML documents or Jupyter notebooks. If an output format supports the format of the raw elements, it might be able to directly use its contents.

$ $

\noindent
\lstset{language=[5.3]Lua}
\begin{lstlisting}
function RawInline(format, str)
  return "\\pandocRawInline{" .. format .. "}{" .. str .. "}"
end
\end{lstlisting}

$ $

\noindent
In the case of our plain \TeX{} writer, it is sensible to directly typeset raw elements of the \texttt{tex} format. Raw elements of other formats will simply be discarded, as seems to be the custom in other Pandoc writers.

Here, as programming of \TeX{} macros is not my forte, expl3 syntax makes an appearance in \file{p2md.tex}. expl3 redefines how some characters are processed. For example, underscores and colons are valid characters inside identifiers of expl3 functions. The significance of whitespace is also reduced. Because of this, all expl3 code has to be surrounded by \macro{ExplSyntaxOn} and \macro{ExplSyntaxOff}.

As for \macro{tl-if-eq-nnT}\todo{find a way to typeset underscores and colons here}, it is a variant of the macro \macro{tl-if-eq-nnTF}, which takes four parameters and returns the third, if the first two are equal, or the fourth, if they are not. \macro{tl-if-eq-nnT} is a variant that completely omits the ``false'' branch and outputs nothing if the elements are not equal, so it only takes three parameters. I have used the indentation style recommended by the expl3 manual\todo{cite}, so snippets using expl3 syntax will be broken into multiple lines.

$ $

\noindent
\lstset{language=[plain]TeX}
\begin{lstlisting}
\ExplSyntaxOn
\def\pandocRawInline#1#2
  {
    \tl_if_eq:nnT
      { #1 }
      { tex }
      { #2 }
  }
\ExplSyntaxOff
\end{lstlisting}

$ $

\noindent
\texttt{Link}, \texttt{Image}, \texttt{CaptionedImage}.\todo{write more}

$ $

\noindent
\lstset{language=[5.3]Lua}
\begin{lstlisting}
function Link(s, tgt, tit, attr)
  return "\\pandocLink{" .. s .. "}{" .. escape(tgt) .. "}{" .. tgt .. "}{" .. escape(tit) .. "}"
end

function Image(s, src, tit, attr)
  return "\\pandocImage{" .. s .. "}{" .. escape(src) .. "}{" .. src .. "}{" .. escape(tit) .. "}"
end

function CaptionedImage(src, tit, s, attr)
  return "\\pandocCaptionedImage{" .. s .. "}{" .. escape(src) .. "}{" .. src .. "}{" .. escape(string.sub(tit, 5)) .. "}"
end
\end{lstlisting}

$ $

\noindent
At the moment, the macros for these elements simply pass their parameters to the corresponding macros of the Markdown package, albeit in greater numbers than before:

$ $

\noindent
\lstset{language=[plain]TeX}
\begin{lstlisting}
\def\pandocLink#1#2#3#4{\markdownRendererLink{#1}{#2}{#3}{#4}}
\def\pandocImage#1#2#3#4{\markdownRendererImage{#1}{#2}{#3}{#4}}
\def\pandocCaptionedImage#1#2#3#4{\markdownRendererImage{#1}{#2}{#3}{#4}}
\end{lstlisting}

$ $

\noindent
The last two inline elements are \texttt{Note} and \texttt{Span}, representing footnotes and HTML span of texts respectively. Again, the attributes of \texttt{Span} are discarded for now, which is unfortunate, since attributes are the main reason of using \texttt{<span>} in HTML in the first place.

$ $

\noindent
\lstset{language=[5.3]Lua}
\begin{lstlisting}
function Note(s)
  return "\\pandocNote{" .. s .. "}"
end

function Span(s, attr)
  return "\\pandocSpan{" .. s .. "}"
end
\end{lstlisting}

$ $

\noindent
\file{p2md.tex}:

$ $

\noindent
\lstset{language=[plain]TeX}
\begin{lstlisting}
\def\pandocNote#1{#1}%
\def\pandocSpan#1{#1}%
\end{lstlisting}

$ $

\subsection{Block elements}

Block elements differ from inline elements mainly in the spacing between them. In a custom writer, it is sufficient to define this spacing just once through the \texttt{Blocksep} function:

$ $

\noindent
\lstset{language=[5.3]Lua}
\begin{lstlisting}
function Blocksep()
  return "\\pandocBlocksep\n"
end
\end{lstlisting}

$ $

\noindent
\file{p2md.tex}:

$ $

\noindent
\lstset{language=[plain]TeX}
\begin{lstlisting}
\def\pandocBlocksep{\markdownRendererInterblockSeparator}%
\end{lstlisting}

$ $

\noindent
The first two block elements, \texttt{Plain} and \texttt{Para}, simply describe blocks of plain text. The main difference (between the two, as well as between them and \texttt{Str}) is in which elements they are allowed to exist as children\todo{check}, but their appearance is virtually the same.

$ $

\noindent
\lstset{language=[5.3]Lua}
\begin{lstlisting}
function Plain(s)
  return s
end

function Para(s)
  return s
end
\end{lstlisting}

$ $

\noindent
Next is \texttt{Lineblock}, a block of lines without indentation (unlike a chain of paragraphs). Since there is no support for blocks of lines in the Markdown package and we would like to maintain the separation between using \pandoc{...} macros in \file{p2md.lua} and more primitive macros in \file{p2md.tex}, the best solution for now is to intersperse the lines with ``hard'' line breaks, which add no indentation.

$ $

\noindent
\lstset{language=[5.3]Lua}
\begin{lstlisting}
function LineBlock(ls)
  return table.concat(ls, "\\pandocLineBreak\n")
end
\end{lstlisting}

$ $

\noindent
There is also \texttt{CodeBlock}, a block version of \texttt{Code}. The crucial difference between these two elements is in the macros that the Markdown package provides -- while \renderer{CodeSpan} simply takes a string of code, which it directly typesets, \renderer{Verbatim} takes the name of a file that containins the block of code. (Additionally, \renderer{FencedCode} takes one more parameter that describes the language of that block of code.) However, in Pandoc's AST, the contents of blocks of code are represented directly inside the \texttt{CodeBlock} elements.

To bridge this gap, the function \texttt{Codeblock} in the writer creates a file inside a helper\file{helper?} directory (see section 3.3) and writes the contents of the codeblock into it. A global variable that counts these blocks is incremented every time this function runs, resulting in unique filenames:

$ $

\noindent
\lstset{language=[5.3]Lua}
\begin{lstlisting}
-- counter for filenames
local codeBlocks = 1

function CodeBlock(s, attr)
  local filename = meta.auxDir .. "/CodeBlock." .. codeBlocks
  local file = assert(io.open(filename, "w"))
  file:write(s)
  file:close()
  codeBlocks = codeBlocks + 1
  return "\\pandocCodeBlock{" .. filename .. "}{" .. attr.class .. "}"
end
\end{lstlisting}

$ $

\noindent
In \file{p2md.tex}, another expl3 macro is used, which determines if the provided language string (obtained from attribute \texttt{class}, this being the one case where we actually do not discard attributes) is empty or not. If not, \renderer{FencedCode} is used instead of \renderer{Verbatim}.
The same block of expl3 syntax also contains a definition of \pandoc{RawBlock} used in the \texttt{RawBlock} function, which is virtually identical to \texttt{RawInline} (see section 3.1.1).

$ $

\noindent
\lstset{language=[plain]TeX}
\begin{lstlisting}
\ExplSyntaxOn
\def\pandocCodeBlock#1#2
  {
    \tl_if_empty:nTF
      { #2 }
      { \markdownRendererInputVerbatim { #1 } }
      { \markdownRendererInputFencedCode { #1 } { #2 } }
  }
\def\pandocRawBlock#1#2
  {
    \tl_if_eq:nnT
      { #1 }
      { tex }
      { #2 }
  }
\ExplSyntaxOff
\end{lstlisting}

$ $

\noindent
An interesting pattern appears in \texttt{Blockquote}, as well as several other block elements. Instead of a single macro, the Markdown package implements two macros, \renderer{BlockQuoteBegin} and \renderer{BlockQuoteEnd}, which are used before and after the contents of the block quote. This was done mainly to avoid memory overflows during macro expansion. The plain \TeX{} writer copies this pattern and implements its own pair of block quote macros.

$ $

\noindent
\lstset{language=[5.3]Lua}
\begin{lstlisting}
function BlockQuote(s)
  return "\\pandocBlockQuoteBegin\n" .. s .. "\n\\pandocBlockQuoteEnd"
end
\end{lstlisting}

$ $

\noindent
\file{p2md.tex}:

$ $

\noindent
\lstset{language=[plain]TeX}
\begin{lstlisting}
\def\pandocBlockQuoteBegin{\markdownRendererBlockQuoteBegin}%
\def\pandocBlockQuoteEnd{\markdownRendererBlockQuoteEnd}%
\end{lstlisting}

$ $

\noindent
The pattern appears again in macros for ordered and unordered lists, although here the macros are nested in two layers, with pairs of \renderer{(Ol/Ul)\-Begin} and \renderer{(Ol/Ul)\-End} surrounding a sequence of pairs of \renderer{(Ol/Ul)\-Item\-Begin} and \renderer{(Ol/Ul)\-Item\-End}, which contain the individual list items.

$ $

\noindent
\lstset{language=[5.3]Lua}
\begin{lstlisting}
function OrderedList(items)
  local buffer = {}
  for _, item in pairs(items) do
    table.insert(buffer, "\\pandocOrderedListItemBegin\n" .. item .. "\n\\pandocOrderedListItemEnd")
  end
  return "\\pandocOrderedListBegin\n" .. table.concat(buffer, '\n') .. "\n\\pandocOrderedListEnd"
end

function BulletList(items, num, sty, delim) -- TODO handle num, sty, delim
  if #items == 0 then
    return ""
  end

  local buffer = {}
  for _, item in pairs(items) do
    table.insert(buffer, "\\pandocBulletListItemBegin\n" .. item .. "\n\\pandocBulletListItemEnd")
  end
  return "\\pandocBulletListBegin\n" .. table.concat(buffer, '\n') .. "\n\\pandocBulletListEnd"
end
\end{lstlisting}

$ $

\noindent
\file{p2md.tex}:

$ $

\noindent
\lstset{language=[plain]TeX}
\begin{lstlisting}
\def\pandocOrderedListBegin{\markdownRendererOlBegin}%
\def\pandocOrderedListItemBegin{\markdownRendererOlItem}%
\def\pandocOrderedListItemEnd{\markdownRendererOlItemEnd}%
\def\pandocOrderedListEnd{\markdownRendererOlEnd}%
\def\pandocBulletListBegin{\markdownRendererUlBegin}%
\def\pandocBulletListItemBegin{\markdownRendererUlItem}%
\def\pandocBulletListItemEnd{\markdownRendererUlItemEnd}%
\def\pandocBulletListEnd{\markdownRendererUlEnd}%
\end{lstlisting}

$ $

\noindent
Definition lists ramp up this complexity once more, this time with a third layer of macros. Inside  each item of the list, which also takes a term as a parameter, one or more definitions for this term can occur -- again, each of them inside a pair of macros.

$ $

\noindent
\lstset{language=[5.3]Lua}
\begin{lstlisting}
function DefinitionList(items)
  local buffer = {}
  for _,item in pairs(items) do
    local term, defs = next(item)
    table.insert(buffer, "\\pandocDefinitionListItemBegin{" .. term .. "}")
    for _,def in pairs(defs) do
        table.insert(buffer, "\\pandocDefinitionListDefinitionBegin\n" .. def ..
                       "\n\\pandocDefinitionListDefinitionEnd{}")
    end
    table.insert(buffer, "\\pandocDefinitionListItemEnd{}")
  end
  return "\\pandocDefinitionListBegin\n" .. table.concat(buffer) .. "\n\\pandocDefinitionListEnd{}"
end
\end{lstlisting}

$ $

\noindent
\file{p2md.tex}:

$ $

\noindent
\lstset{language=[plain]TeX}
\begin{lstlisting}
\def\pandocDefinitionListBegin{\markdownRendererDlBegin}%
\def\pandocDefinitionListItemBegin#1{\markdownRendererDlItem{#1}}%
\def\pandocDefinitionListDefinitionBegin{\markdownRendererDlDefinitionBegin}%
\def\pandocDefinitionListDefinitionEnd{\markdownRendererDlDefinitionEnd}%
\def\pandocDefinitionListItemEnd{\markdownRendererDlItemEnd}%
\def\pandocDefinitionListEnd{\markdownRendererDlEnd}%
\end{lstlisting}

$ $

\noindent
For document headings, the Markdown package has six macros, \renderer{HeadingOne} to \renderer{HeadingSix}. Design-wise, this might make sense as a mapping to \LaTeX{} macros like \macro{chapter}, \macro{section}, \macro{subsection} etc. However, the corresponding Pandoc element simply passes the heading level as an integer.

$ $

\noindent
\lstset{language=[5.3]Lua}
\begin{lstlisting}
function Header(lev, s, attr)
  return "\\pandocHeader{" .. lev .. "}{" .. s .. "}"
end
\end{lstlisting}

$ $

\noindent
Inside \file{p2md.tex}, the expl3 macro \macro{int-case-nn}\todo{find a way to typeset underscores and colons here} simply picks the right one of these macros depending on the level.

$ $

\noindent
\lstset{language=[plain]TeX}
\begin{lstlisting}
\ExplSyntaxOn
\def\pandocHeader#1#2
  {
    \int_case:nn
      { #1 }
      {
        { 1 } { \markdownRendererHeadingOne { #2 } }
        { 2 } { \markdownRendererHeadingTwo { #2 } }
        { 3 } { \markdownRendererHeadingThree { #2 } }
        { 4 } { \markdownRendererHeadingFour { #2 } }
        { 5 } { \markdownRendererHeadingFive { #2 } }
        { 6 } { \markdownRendererHeadingSix { #2 } }
      }
  }
\ExplSyntaxOff
\end{lstlisting}

$ $

\noindent
\texttt{HorizontalRule} simply represents a horizonal rule, while \texttt{Div} represents a \texttt{<div>} element from HTML. As with \texttt{Span} before, attributes are discarded for now.

$ $

\noindent
\lstset{language=[5.3]Lua}
\begin{lstlisting}
function HorizontalRule()
  return "\\pandocHorizontalRule"
end

function Div(s, attr)
  return "\\pandocDivBegin\n" .. s .. "\n\\pandocDivEnd{}"
end
\end{lstlisting}

$ $

\noindent
\file{p2md.tex}:

$ $

\noindent
\lstset{language=[plain]TeX}
\begin{lstlisting}
\ExplSyntaxOn
\def\pandocHorizontalRule{\markdownRendererHorizontalRule}%
\def\pandocDivBegin{}%
\def\pandocDivEnd{}%
\end{lstlisting}

$ $

\noindent
Finally, \texttt{Table}. This one do be kinda fucked up tho.\todo{elaborate}

$ $

\noindent
\lstset{language=[5.3]Lua}
\begin{lstlisting}
function Table(caption, aligns, widths, headers, rows)
  local coldef = ""
  for col,def in pairs(aligns) do
    if     def == "AlignLeft"    then coldef = coldef .. "l"
    elseif def == "AlignRight"   then coldef = coldef .. "r"
    elseif def == "AlignDefault" then coldef = coldef .. "c"
    elseif def == "AlignCenter"  then coldef = coldef .. "d"
    end
  end

  local buffer = {"\\pandocTable{"}
  if caption ~= nil then table.insert(buffer, caption) end
  table.insert(buffer, "}{" .. #rows .. "}{" .. #aligns .. "}{" .. coldef .. "}")

  for i,row in ipairs(rows) do
    local row_buffer = {"{"}
    for j,cell in ipairs(row) do
      table.insert(row_buffer, "{" .. cell .. "}")
    end
    table.insert(row_buffer, "}")
    table.insert(buffer, table.concat(row_buffer))
  end

  return table.concat(buffer)
end
\end{lstlisting}

$ $

\noindent
And in \file{p2md.tex}:

$ $

\noindent
\lstset{language=[plain]TeX}
\begin{lstlisting}
\ExplSyntaxOn
\def\pandocTable{\markdownRendererTable}%
\end{lstlisting}

$ $

\section{Other functions}
\emph{\texttt{Doc}, \texttt{meta}}

\section{\LaTeX{} macropackage surrounding the writer}
\emph{Shortly describe \file{p2md.sty} and the overrides that happen here.}
\subsection{Inline elements}
\subsection{Block elements}

\section{Example documents}
\emph{Demonstrate specific example documents and the problems they posed while implementing the writer. \label{fig:html-browsers-typeset} \todo{Consider if this should not be a separate chapter.}}

% \includepdf[pages={1}]{pandoc-to-markdown/examples/html-browsers.pdf}

% \begin{figure}[h]
%    \centering
%    \begin{tabular}{@{}c@{\hspace{.5cm}}c@{}}
%      \includegraphics[
%        width=0.6\textwidth,
%        page=1,
%        frame,
%        trim=2cm 2cm 2cm 2cm
%      ]{pandoc-to-markdown/examples/html-browsers.pdf}
%    \end{tabular}
%  \caption{Example page from an \textsc{HTML} document typeset with the proof of concept}
%  \label{fig:html-browsers-typeset}
% \end{figure}

\chapter{Conclusion}
\section{Future work}
\emph{Mention open issues. Also the Haskell plans, solidification of the macros interface and the relevant ongoing discussion.}

\subsection{Handling of deprecation of classic custom writers}

There are currently two styles of defining custom writers for Pandoc, to which the Pandoc documentation refers to as ``classic'' and ``new''. The main difference between them is that the classic style writer defines its rendering functions on the top level, while in the new style, everything is processed through a top-level function named \texttt{Writer} which contains all the custom writer logic.

For the plain \TeX{} writer, I have decided to go with the classic style. There were multiple reasons for this, the main one being that the classic style is what Pandoc's documentation covers. Another reason is that support for the new style was only introduced in Pandoc 2.18\footnote{The Pandoc documentation actually states that the new style of custom writers was introduced in 2.17.2. This is wrong. No such version was even released, 2.17.1.1 was immediately followed by 2.18.}, which was released in January 2022. Repositories of several high-profile Linux distributions ship versions of Pandoc that are way older than that. For instance, Fedora 37 released in November 2022 provides only packages with Pandoc 2.14.0.3\footnote{\url{https://packages.fedoraproject.org/pkgs/pandoc/pandoc/fedora-37.html}}, which was released in June 2021. Another example is Ubuntu 22.04 LTS (codenamed ``Jammy Jellyfish''), which, despite being released in April 2022, still ships Pandoc 2.9.2.1\footnote{\url{https://packages.ubuntu.com/jammy/pandoc}} from March 2020.

Unbeknownst to me at the time, classic custom writers were going to be deprecated. According to the discussion on the related pull request\footnote{\url{https://github.com/jgm/pandoc/pull/8343}}, it will still be possible to update a classic writer to a new-style one. However, other breaking changes might still be introduced. As long as the plain \TeX{} writer is an external one instead of being built into Pandoc itself, it will have to deal with this in the future.

\printbibliography[heading=bibintoc] %% Print the bibliography.

\makeatletter\thesis@blocks@clear\makeatother
\phantomsection %% Print the index and insert it into the
\addcontentsline{toc}{chapter}{\indexname} %% table of contents.
\printindex

\appendix %% Start the appendices.
\chapter{An appendix}
Here you can insert the appendices of your thesis.

\end{document}
